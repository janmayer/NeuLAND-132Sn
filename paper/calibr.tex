\documentclass{scrartcl}
\usepackage{tikz}
\usepackage{graphicx}

\begin{document}

\begin{figure}
\begin{tikzpicture}% [scale=\columnwidth/252.0pt]
\node [anchor=north west,inner sep=0] (img) at (0,0) {\includegraphics[width=\textwidth]{{15m_30dp_600AMeV_500keV_5n.ncut}.pdf}};
\node at (2,-0.4) {1n};
\node at (4.922,-0.4) {2n};
\node at (7.844,-0.4) {3n};
\node at (10.766,-0.4) {4n};
\node at (13.688,-0.4) {5n};
\node [left, rotate=90] at (0,0)  {\small \# Clusters};
\node [left] at (14.5,-2.8) {\small Total Deposited Energy $E_{dep}$};
\end{tikzpicture}
\caption{Calorimetric approach to distinguish neutron multiplicities.
The total deposited energy is shown together with the number clusters for neutrons with a kinetic energy of 600\,MeV.
In the simulated data for 1 to 5 neutrons, cuts are set to separate the different multiplicities.\label{fig:calibr}}
\end{figure}

\end{document}